\documentclass{article}
\usepackage[utf8]{inputenc}
\usepackage{amstext}
\usepackage{amsmath}
\usepackage{amsfonts}
\usepackage{graphicx}
\usepackage[margin=1in, paperwidth=8.5in, paperheight=11in]{geometry}
\usepackage{gensymb}
\usepackage{indentfirst}
\usepackage{textcomp}
\usepackage{upgreek }
\usepackage{siunitx}
\usepackage{enumitem}

\usepackage[american]{circuitikz}

\title{Circuits Postlab 8}
\author{Byron Wasti}
\date{April 2017}

\begin{document}
\maketitle

\section{Problem 1}

When analyzing what happens to the circuit in the first experiment, we can break it up into three portions: $V_1 > V_2$, $V_1 \approx V_2$ and $V_1 < V_2$. We will analyze each of these pieces seperately.

\begin{itemize}

    \item [($V_1 < V_2$):]
        First, we can ignore the region where $V_1 < V_b$, since neither $M_1$ or $M_2$ will be in saturation, and the circuit will not pass any current. 
        
        In the region where $V_b < V_1 < V_2$, however, $V_{out}$ cannot drop below approximately $\kappa(V_1 -V_b)$. This is due to the fact that $M_1$ and $M_2$ pass the same amount of current, and $M_1$ in parallel with $M_2$ act as a source-follower with $M_b$. Thus, we know that $V \approx \kappa(V_1 - V_b)$ since the current through both $M_2$ and $M_1$ is controlled by $V_1$.

        If $V_{out}$ were to go below $V$, and thus go below approximately $\kappa(V_1 - V_b)$, then $M_2$ would have current flowing into the $V_{out}$ node (reverse of the normal flow of current). Since $M_3$ and $M_4$ are current mirrors, and both in saturation, then there would also be current flowing into $V_{out}$ from $M_4$. Since there is no place for the current to go, $V_{out}$ would increase until $M_2$ is no longer passing current in the wrong direction.

        Thus, $V_{out}$ cannot go below $V$, and since $V$ is approximately $\kappa(V_1 - V_b)$, $V_{out}$ cannot go below $\kappa(V_1 - V_b)$. This can be seen in our experimental data as the linear region of the plot when $V_1 < V_2$.

    \item [($V_1 \approx V_2$):] 
        When $V_1 \approx V_2$ we get the effects of the differential-mode voltage gain, which is infinite if we do not take into account the Early effect. As we found experimentally, this differential-mode gain is not infinite due to the Early effect. In general, in this region, $V_{out} = A_{dm} V_{dm}$. Thus, slight changes in $V_{dm}$ cause large swings in $V_{out}$ as shown in the experimental data.

    \item [($V_1 > V_2$):] 
        This is the simplest region to analyze, which is when $V_1 > V_2$ at steady state. In this case, almost all of the current in the circuit is going through the left branch of the differential amplifier. This is because since $V_1 > V_2$, there is a larger current flowing through $M_1$ which is mirrored due to the $p$MOS current mirror. Thus, there is an excess current at the drain node of $M_2$, raising the node capacitance until $M_4$ is no longer in saturation.

\end{itemize}

\end{document}

