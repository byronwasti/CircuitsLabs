\documentclass{article}
\usepackage[utf8]{inputenc}
\usepackage{amstext}
\usepackage{amsmath}
\usepackage{amsfonts}
\usepackage{graphicx}
\usepackage[margin=1in, paperwidth=8.5in, paperheight=11in]{geometry}
\usepackage{gensymb}
\usepackage{indentfirst}
\usepackage{textcomp}
\usepackage{upgreek }
\usepackage{siunitx}
\usepackage{enumitem}

\usepackage[american]{circuitikz}

\title{Circuits Postlab 5}
\author{Byron Wasti}
\date{March 2017}

\begin{document}
\maketitle

\section{nMOS}

\begin{itemize}
    \item [Strong 0: ] A strong 0 for an nMOS transistor is essentially looking at the drain characteristics of an nMOS transistor when the drain is connected to a node-capacitance. Our data from Experiment 3 shows us the drain characteristics of an nMOS transistor, which shows that the current continues to decay exponentially as $V_{drain}$ gets closer to $0V$. This means that the voltage difference between $V_{source}$ (which is at $0V$ by definition) and $V_{drain}$ continues to get exponentially smaller. This is why an nMOS produces a strong 0.


    \item [Weak 1: ]  On the flipside, a weak 1 can be analyzed by looking at an nMOS's source characteristics. We did this in Experiment 2, and found that the current seemed to reach a constant value when the source voltage reached $4V$ to $5V$. Since there is current flow, that means there is a voltage difference between $V_{drain}$ (which is $5V$ by definition) and $V_{source}$. Thus, an nMOS produces a weak 1.

\end{itemize}

\section{pMOS}

\begin{itemize}
    \item [Weak 0: ] For a pMOS transistor, we can analyze why it produces a weak 0 by looking at the source charcateristics of the pMOS transistor. From Experiment 2, we see that the current reaches a constant value once the pMOS transistor's source voltage gets close to 0V. This means that there is a voltage difference between $V_{drain}$ (which is $0V$ by definition) and $V_{source}$, which means that $V_{source}$ is a weak 0.

    \item [Strong 1: ] A strong 1 for a pMOS transistor can be seen due to the analysis of the drain characteristics of a pMOS transistor, which we did in Experiment 3. The current continues to decay exponentially as $V_{drain}$ gets closer to $5V$, which means that the voltage difference between $V_{source}$ ($5V$ by definition) and $V_{drain}$ continues to decrease. Thus, the pMOS is able to produce a strong 1.

\end{itemize}

\end{document}
